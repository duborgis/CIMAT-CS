

{\large Direções Cristalinas Slide 04}

\section{Direções Cristalográficas}

\begin{itemize}
	\item Propriedades direcionais: Ex módulo de elasticidade do Fe maior na diagonal do cubo do que na aresta.
	\item Direções cristalinas são indexadas através de um segmento que se estende da origim até determinada posição dentro do cristal.
	\item Três índices são usados para designação das direções:

\end{itemize}


\begin{itemize}
	\item Índices de Müller$\rightarrow$ Base: célula unitária com sistema de coordenadas cartesianas, com origem em um vértice da célula.
	\begin{itemize}
		\item Vetor posicionado de forma a passar pela origem do sistema de coordenadas cartesianas. Pode ser transladado, desde que seja mantido um paralelismo com o vetor na origem.
		\item O comprimento da projeção do vetor em cada eixo é determinado em relação aos parâmetros de rede a, be c.
		\item Redução dos valores ao menor número inteiro por multiplicação ou divisão por um fator comum.
		\item Os três índices não são separados por vírgulas e ficam entre colchetes. [u v w]
		\item [u v w] correspondem às projeções nos eixos x, y e z, respectivamente.
	 	\item Em cristais cúbicos, as direções do tipo [111] (índices negativos e positivos) são iguais e identificadas por <111>.
	\end{itemize}	
\end{itemize}

{\large no slide 04 podemos achar alguns exemplos do índice de Müller para sólidos}


\section{Densidade de Sólidos}





\begin{itemize}
	\item \textbf{Densidade volumétrica}: $D_{v}$= massa por célula unitária / volume da célula unitária
	\item \textbf{Densidade Linear}: $D_{l}$= átomos centrados sobre o vetor direção / comprimento do vetor direção
\end{itemize}
