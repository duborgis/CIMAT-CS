\section{Evolução dos materiais}
\subsection*{Idade da Pedra}
Materiais mais comuns: Polímeros (madeiram peles e fibras);Cerâmicos (Pedra)
\subsection*{Idade da Argila}
Materiais mais comuns: 
\subsection*{Idade do Cobre}
Materiais mais comuns: 
\subsection*{Idade do Bronze}
Materiais mais comuns: 
\subsection*{Idade do Ferro}
Materiais mais comuns: 
\section{Sociedade e Materiais}
Aumento significativo de tipos de materiais com o desenvolvimento industrial $\rhd$ Melhoria nos métodos de extração e produção \& Alterações nos materiais para formação de novos materiais (materiais avançados, compósitos)
\section{Classificação dos Materiais}
\subsection*{Metais}
Combinação de elementos metálicos, sendo possível a presença de não-metálicos (ex: aço-carbono).Algumas propriedades relacionadas à presença de elétrons livres (ex: bons condutores de calor e eletricidade).Resistentes e deformáveis: extenso uso em aplicações estruturais.
Suscetível a corrosão.

Possui boa condutividade térmica e elétrica.
\subsection*{Cerâmicos}
ligações de metais e não-metais. Isolantes térmicos e elétricos. Podem ser resistentes a altas temperaturas e a ambientes agressivos. Ex: óxidos, nitretos, carbetos.
Resistentes a corrosão.

Possui boa condutividade térmica e variada condutividade elétrica.

\subsection*{Polímeros}


polímeros:compostos orgânicos. Baixa densidade; altamente deformáveis. (C, H e não-metálicos)
Degrada com solventes, altas temperaturas.

Possui baixa condutividade térmica e elétrica.

\subsection*{Compósitos}

Compósitos: dois ou mais tipos de materiais unidos de forma a produzir um material com características específicas. Projetados para apresentar uma combinação das melhores características de cada um dos componentes.

\subsection*{Materiais avançados}

Uso em aplicações de ponta(alta tecnologia). Não necessariamente são novos materiais. Podem ser tradicionais, com otimização das propriedades.

\begin{itemize}
		
	\setlength{\parskip}{0pt}
	\setlength{\itemsep}{0pt plus 1pt}
	
\item Alto desempenho
\item Baixo peso e alta resistência
\item Resistência a diversas condições de serviço
\item Ambientalmente corretos
\item Facilmente recicláveis
\end{itemize}
Ex: polímero reforçado com fibra de vidro usado como armadura para estruturas em concreto armado


\section{Classificação dos Materiais}

\subsection*{Qaunto a Estrutura}

\begin{itemize}
		
	\setlength{\parskip}{0pt}
	\setlength{\itemsep}{0pt plus 1pt}
	
	\item Ligações atômicas
	
	\begin{itemize}
			
		\setlength{\parskip}{0pt}
		\setlength{\itemsep}{0pt plus 1pt}
		
		\item Metálica
		\item covalente
		\item iônica
	\end{itemize}

	\item Tipo de Estrutura
	\begin{itemize}
			
		\setlength{\parskip}{0pt}
		\setlength{\itemsep}{0pt plus 1pt}
		
		\item Amorfa
		\item Cristalina
		\item Molecular
	\end{itemize}
\end{itemize}


\subsection*{Propriedades Mecânicas}
\begin{itemize}
		
	\setlength{\parskip}{0pt}
	\setlength{\itemsep}{0pt plus 1pt}
	
	\item dutilidade
	\item elasticidades
	\item dureza
	\item tenacidade
\end{itemize}

\subsection*{Propriedades Químicas e Físicas}
\begin{itemize}
		
	\setlength{\parskip}{0pt}
	\setlength{\itemsep}{0pt plus 1pt}
	
\item	ponto de fusão
\item	calor específico
\item	condutividade (térmica e elétrica)
\item 	Propriedades magnéticas
\item 	Propriedades ópticas
\item 	Iteração com o ambiente: oxidação 
\end{itemize}

\subsection*{Modificação das Propriedades}
\begin{itemize}
		
	\setlength{\parskip}{0pt}
	\setlength{\itemsep}{0pt plus 1pt}
	
	\item Tratamentos térmicos
	\item tratamento de superfície
	\item Tratamentos mecânicos
\end{itemize}


\section{COMPOSIÇÃO QUÍMICA x PROPRIEDADES FÍSICAS}
Exemplo 1: alteração de cor de um mesmo mineral

CORÍNDON:mineral à base de óxido de alumínio (Al2O3)
Chama-se safira qualquer variedade de corindon, de qualidade gemológica, que não seja vermelha (rubi)

Exemplo 2: Alteração da compatibilidade química com o meio ambiente.
Aço patinável (CORTEN): aço com adição de cobre que forma uma pátina, reduzindo o processo corrosivo. Muito usado na construção civil.



\section{ESTRUTURA CRISTALINA x PROPRIEDADES FÍSICAS}
CaCO3

Mármore (calcita -romboédrico)

Conchas (aragonita -ortorrômbico)

\section{TIPO DE LIGAÇÃO QUÍMICA x PROPRIEDADES FÍSICAS}

O tipo de ligação iônica está intimamente ligada com as características físico-química dos materiais.

Temos que:
\subsection*{Ligações Metálicas}
Favorecem a condutividade térmica e elétrica

\subsection*{Ligações Covalentes}
Aparecem em matérias com características isolantes. 


\section{Materiais de Construção}

\subsection*{Escolha do Material}

Etapas básicas

\begin{itemize}
	
\setlength{\parskip}{0pt}
\setlength{\itemsep}{0pt plus 1pt}

\item relacionar experiências prévias para o serviço em questão
\item relacionar e colocar em ordem de prioridade todos os parâmetros que podem influenciar na escolha
\item estabelecer as características que se deseja para o material ideal ao serviço
\item realizar testes, comparando os materiais que possam ser usados otimizando o custo total.

\end{itemize}

Definição da categoria do material: metal, cerâmico, polímero ou compósito

\subsection*{Como selecionar um material?}

Condições operacionais
\begin{itemize}
	
	\setlength{\parskip}{0pt}
	\setlength{\itemsep}{0pt plus 1pt}
	
\item Custo/benefício (indústria de massa ou de grande exigência tecnológica?)
\item Propriedades requeridas
\item Degradação no meio
\item Solicitação mecânica
\item Resistência/peso
\end{itemize}

\subsection*{Quais fatores influenciam na escolha dos materiais para determinado serviço ou aplicação industrial?}

\begin{itemize}
	\setlength{\parskip}{0pt}
	\setlength{\itemsep}{0pt plus 1pt}
	
	\item Propriedades físicas e químicas 
	\item Características operacionais
	\item Fabricação, disponibilidade e custo
	
\end{itemize}


\section{Propriedades físicas e químicas}

\subsection*{Propriedades mecânicas}
Limites de escoamento e de resistência, dutilidade, resistência à fadiga e fluência, temperatura de transição dútil-frágil, módulo de elasticidade.

\subsection{Outras propriedades físicas}
densidade, calor específico,expansão térmica,condutividade,propriedades magnéticas e elétricas.


\subsection*{Propriedades químicas}
oxidação, flamabilidade,toxicidade.

\section{Características operacionais}

\subsection*{temperatura}

\subsection*{natureza dos fluidos em contato com material}

(composição química, concentração, pH, caráter oxidante ou redutor, presença de contaminantes, pressão, velocidade do fluido, flamabilidade, toxidez). Necessário avaliar alterações em cada um destes fatores ao longo do tempo de serviço.

\subsection*{presença de resíduos oriundos de processos corrosivos}
ex: alguns materiais como o chumbo são resistentes à corrosão porém geram resíduos tóxidos, limitando seu uso.

\section{Fabricação,disponibilidade e custo}


\subsection*{Fatores críticos na seleção:} facilidade de fabricação e forma de apresentação. Ex:espessura de chapas e formato(tubos ou tarugos)
Custo/benefício: custo direto do material, tempo de vida útil e custos indiretos (paradas para reparos ou reposição dos equipamentos).
Tempo de vida útil compatível com tempo previsto de operação.
Relação direta com segurança (acidentes e falhas)
Furos em tubulações.
Aço inox 304 no lugar de aço-C.
Aço-carbono disponível em inúmeras formas de fabricação.
